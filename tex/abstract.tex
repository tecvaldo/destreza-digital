\begin{abstract}
	Este estudo tem como objetivo analisar a diferença salarial e a diferença na quantidade de demissões entre homens e mulheres na área de Tecnologia da Informação (TI) no Brasil no período de 2015 a 2021. Para isso, foi realizada uma pesquisa bibliográfica sobre o tema, além de uma análise de dados da Relação Anual de Informações Sociais (RAIS) do Ministério da Economia. Os resultados mostraram que a quantidade de mulher na área de TI equivale a, no máximo, 30\% da quantidade de homens. Além disso, foi verificado que em todos os anos de estudo há uma diferença salarial em desfavor da mulher e independente do nível escolar, do cargo de TI e do setor onde trabalham, se público ou privado. Já com relação à quantidade de demissões, não foi possível perceber uma diferença significativa entre os gêneros, execeto em 2021, quando a porcentagem de demissões dos homens foi de 35,82\% e das mulheres doi de 29,91\%.
\end{abstract}

\begin{IEEEkeywords}
	Transformação Digital. Maturidade digital. Destreza Digital. Mentalidade Digital. Governo. Setor Público. Ministério Público.
\end{IEEEkeywords}
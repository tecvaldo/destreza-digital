\section{Referêncial Teórico e Hipóteses da Pesquisa}

Nos últimos anos, muitos estudos vêm sendo realizados no contexto internacional com o intuito de desmistificar a Transformação Digital e apresentar requisitos ou modelos para sua implementação ou ainda critérios de avaliação da maturidade digital das organizações. Outros focam nos desafios vivenciados pelas organizações para estabelecer uma mentalidade e uma cultura digital, bem como nas taxas de insucesso no estabelecimento e cumprimento de estratégias digitais. Entretando, a CPE entendendo que a TD é uma jornada embricada de desafios complexos, deseja estabelecer um caminho, traçar um alvo e uma rota. Para tanto, é mandatório entender o cenário atual para iniciar a marcha rumo a um MP Digital. Com esse propósito, a CPE realizou um diagnóstico da maturidade digital do MP Brasileiro, tendo como base o trabalho de Rossmann (2018), que efetuou uma compilação abrangente de vários estudos visando desenvolver um modelo de avaliação para a maturidade digital. Neste estudo, foi destacado que a maturidade digital abrange oito dimensões essenciais: estratégia, liderança, mercado, operações, recursos humanos, cultura organizacional, governança e tecnologia. Essas dimensões foram adaptadas à realidade do Ministério Público e aos requisitos preconizados pela Política Nacional de Tecnologia da Informação (PNTI) \footnote{PNTI - \url{https://www.cnmp.mp.br/portal/atos-e-normas-busca/norma/5189}} e estão definidas a seguir.

\subsection{Estratégia Digital da Organização }

A estratégia digital representa um componente crucial para organizações que desejam se destacar na era digital, oferecendo um roteiro compreensivo para a incorporação da tecnologia digital em todos os aspectos organizacionais. Ela ultrapassa os limites do Planejamento Estratégico de TI tradicional \cite{dolganova2019company}, concentrando-se não apenas na otimização das operações de TI, mas na utilização holística da tecnologia para transformar relações com clientes, processos internos e para buscar novas oportunidades de mercado. A estratégia digital documentada é fundamental, funcionando não somente como um complemento do planejamento estratégico, mas como um guia detalhado que destaca metas e aborda os desafios específicos do ambiente digital. Este aspecto garante que a organização esteja preparada para enfrentar as complexidades e as evoluções contínuas do cenário tecnológico.

Hipótese 1: O percentual de unidades do Ministério Público Brasileiro com uma estratégia digital definida e documentada, amplamente divulgada e que efetivamente influencia a estratégia da organização é baixo. Esta hipótese aponta para uma possível lacuna na adoção de abordagens estratégicas digitais, o que pode resultar em esforços de transformação digital descoordenados ou ineficazes.

\subsection{Serviços Digitais ao Cidadão}

Como se trata de uma instituição pública sem fins lucrativos, optou-se por substituir a dimensão mercado por serviços digitais ao cidadão. A disponibilização desses serviços ao cidadão é um aspecto fundamental para as organizações públicas na era digital, servindo como um indicador da maturidade digital e do compromisso com a modernização e eficiência dos serviços públicos. Isso está em consonância com a Lei Nº 14.129 \footnote{Lei 14.129 \label{lei14129} - \url{https://www.planalto.gov.br/ccivil_03/_ato2019-2022/2021/lei/l14129.htm}}, de 29 de março de 2021, em seu Art. 3º, define como princípios e diretrizes do Governo Digital e da eficiência pública “a desburocratização, a modernização, o fortalecimento e a simplificação da relação do poder público com a sociedade, mediante serviços digitais, acessíveis inclusive por dispositivos móveis”. Estes serviços não apenas aumentam a eficiência e a agilidade dos processos, mas também promovem a transparência e o engajamento do cidadão, elementos essenciais na construção de uma relação de confiança e responsabilidade entre a instituição pública e a população. A importância de tais serviços é amplificada pela necessidade de avaliar continuamente a satisfação do cidadão, o que também é previsto no inciso V da mesma Lei Nº 14.129 \footref{lei14129}, que diz que as Plataformas de Governo Digital devem oferecer instrumento de “avaliação continuada da satisfação dos usuários em relação aos serviços públicos prestados”. Adicionalmente, a manutenção de um catálogo atualizado de serviços, indicando quais são digitais e quais são analógicos, é um sinal da transparência da instituição e de sua capacidade de autoavaliação em sua jornada de transformação digital.

Hipótese 2: O percentual de unidades do Ministério Público Brasileiro que efetivamente oferecem uma gama diversificada de serviços digitais ao cidadão, possuem mecanismos efetivos de avaliação da satisfação do usuário e mantêm um catálogo atualizado de serviços digitais e analógicos é baixo. Esta hipótese sugere uma lacuna na oferta e no gerenciamento de serviços digitais, potencialmente resultando em uma experiência de usuário subotimizada e em oportunidades perdidas de melhoria na eficiência e transparência dos serviços públicos.

\subsection{Pessoas}

Apesar das ferramentas digitais estarem no centro das conceituações coletivas de transformação digital, a dimensão pessoa continua sendo um pilar nesse processo. Segundo a ITIL® 4 \footnote{ITIL 4 - \url{https://www.axelos.com/certifications/itil-service-management}}, é necessário garantir que as equipes tenham as habilidades, competências e motivação necessárias para entregar serviços digitais de alta qualidade. Esta dimensão se concentra em compreender o compromisso da organização com temas críticos como inovação, transformação digital, governança institucional e governança de TI, avaliando o número de profissionais dedicados exclusivamente ou parcialmente a essas áreas, bem como outros fatores como capacitação contínua e plano de carreira para retenção de talentos. A presença de equipes dedicadas, capacitadas e valorizadas demonstra uma abordagem estratégica e proativa para manter a organização atualizada e eficiente em um ambiente tecnológico em constante mudança.

Hipótese 3: O número de profissionais dedicados exclusiva ou parcialmente a temas como inovação, transformação digital, governança institucional e governança de TI no Ministério Público Brasileiro, assim como a existência de programas de capacitação contínua e planos de carreira específicos para estas áreas, é insuficiente. Esta hipótese sugere uma lacuna na alocação de recursos humanos e no investimento em desenvolvimento profissional, o que pode resultar em uma capacidade limitada de adaptação, inovação e implementação eficaz de estratégias digitais dentro da organização.

 \subsection{Gestão e Governança}

 Se tecnologia é uma habilitadora da transformação digital \cite{transformacaoDigitalKraus2021}, fazer a gestão e governança do seu uso dentro da organização é ação mandatória. Assim sendo, nesta dimensão foi incluído o termo gestão além da governança sugerida por Rossmann \cite{rossmann2018digital} para avaliar a existência de um planejamento estratégico focado na digitalização e automação de processos de trabalho, bem como a efetividade na adoção e monitoramento de tecnologias digitais, metodologias ágeis e boas práticas de mercado. A utilização de boas práticas de mercado na entrega, gestão e governança de serviços demonstra uma busca pela excelência operacional e pela satisfação do usuário além de atender requisitos da Lei Nº 14.129 \footref{lei14129}, Art. 47 que versa sobre a necessidade de se "implementar e manter mecanismos, instâncias e práticas de governança".

 Hipótese 4: O Ministério Público Brasileiro possui boas iniciativas na integração de ações estratégicas de automação e digitalização em seu planejamento estratégico, bem como na adoção efetiva de tecnologias digitais, metodologias ágeis e práticas de mercado na gestão de serviços, mas num nível ainda baixo. Esta hipótese sugere que as unidades podem estar enfrentando desafios para alcançar uma governança digital eficiente, o que pode limitar sua capacidade de responder de forma eficaz às demandas do ambiente digital e tecnológico atual, impactando a eficiência e a qualidade dos serviços prestados.
 
 \subsection{Liderança}

 Segundo Chatterjee \cite{chatterjee2002shaping}, os líderes devem confiar no valor e nos benefícios das novas tecnologias de TI e apoiar a sua implementação. Na pesquisa realizada por Swen Nadkarni \& Reinhard Prugl \cite{nadkarni2021digital}, constatou-se um consenso de que a “alta-gestão necessita de uma nova mentalidade digital para liderar a jornada de transformação digital da sua organização”. Logo, os gestores devem repensar as suas práticas de educação para além de habilidades de liderança e comunicação. Sia et al. \cite{sia2016dbs}, por exemplo, conduziram um estudo de caso sobre um banco asiático que utiliza hackathons para educar os membros da alta-gestão.

 Destarte, esta dimensão avalia a participação da alta-administração em ações de capacitação em inovação e transformação digital como um forte indicador do compromisso da liderança com a modernização em um ambiente digital em constante mudança. Isso demonstra uma abordagem proativa e um exemplo motivador para toda a organização, influenciando positivamente a cultura de inovação. Além disso, o monitoramento de resultados da gestão e adoção de políticas modernas de gestão de trabalhadores, especialmente no que se refere às iniciativas de trabalho remoto, revela uma postura data-driven e uma mentalidade digital por parte da liderança, essencial para impulsionar e sustentar iniciativas digitais bem-sucedidas.

 Hipótese 5: No Ministério Público Brasileiro, a liderança pode não estar suficientemente envolvida ou capacitada em temas de inovação e transformação digital, sendo um entrave para a maturidade digital. Esta hipótese sugere que a instituição pode estar enfrentando desafios para impulsionar uma cultura de inovação e modernização efetiva em razão de lideranças sem a adequada alfabetização digital, o que pode limitar sua capacidade de adaptação e crescimento no cenário digital atual.
 

\subsection{Operações}

Nesta dimensão, Rossmann \cite{rossmann2018digital} foca na construção de parcerias para melhorar a capacidade de inovação das organizações. A política do MP Digital \footnote{Estratégia Digital do Ministério Público - \url{https://www.cnmp.mp.br/portal/atos-e-normas/norma/9724/}} também estimula a colaboração e có-criação de serviço de TI, entretanto, esse tema foi abordado da dimensão "Cultura". Um outro aspecto que Rossmann \cite{rossmann2018digital} aborda aqui é a existência de recursos (tempo, pessoas e orçamento) suficientes. Seguindo nessa linha, o enfoque desta dimensão foi a avaliação da estrutura, recursos e orçamento dedicados à tecnologia da informação e às iniciativas digitais. Isso também está alinhado com a visão de Kontić \& Vidicki \cite{kontic2018strategy} que afirmam que "liderança proativa e investimento são os principais fatores que determinam o potencial de uma empresa para se tornar uma organização digital". Abordou-se, portanto, identificar a proatividade dos líderes na criação de estruturas organizacionais específicas para temas críticos como inovação e transformação digital com profissionais qualificados e valorizados, além de detalhes sobre o orçamento destinado a diferentes aspectos da operação, incluindo gastos com pessoal, custeio, telecomunicações, investimentos em TI, e despesas com arquitetura e engenharia. Esses dados oferecem uma visão sobre a priorização dos investimentos da instituição em tecnologia e revelam sua capacidade de atender às necessidades da TD.

Hipótese 6: O Ministério Público não prioriza a agenda digital de forma efetiva, dotando recursos aquém dos necessários para impulsionar a maturidade digital. Esta hipótese sugere que, sem uma estrutura e alocação de recursos adequados, a instituição pode não estar totalmente preparada para enfrentar os desafios da era digital e pode estar perdendo oportunidades de melhorar a eficiência e eficácia de suas operações.


\subsection{Cultura}

Uma transformação digital bem-sucedida deve ser capaz de envolver cada funcionário da organização. Os funcionários precisam estar energizados, entusiasmados e envolvidos no processo de transformação. Portanto, uma nova cultura digital tem que ser instalada e isso começa com o comportamento dos altos executivos \cite{Hie_2019}. Para tanto, nesta dimensão, optou-se por avaliar a adaptabilidade, a comunicação de transformações, a inclusão de membros e servidores em processos de inovação, o reconhecimento de iniciativas inovadoras e a colaboração interinstitucional. 
A presença de um processo formal de gestão de mudança, a existência de programa para envio de ideias e a comunicação dos casos de sucesso são indicativos de que a organização está preparada para transitar eficientemente para estados desejados, minimizando resistências e maximizando o engajamento. Por fim, a colaboração com outros Ministérios Públicos no desenvolvimento e uso de serviços digitais reflete uma abordagem orientada à comunidade, promovendo a eficiência e o compartilhamento de melhores práticas.

Hipótese 7: O Ministério Público Brasileiro pode estar enfrentando desafios na implementação de uma cultura organizacional adaptativa e inovadora. Esta hipótese sugere que a instituição pode não estar maximizando seu potencial de inovação e adaptação em um ambiente tecnológico em rápida evolução, impactando sua capacidade de promover uma cultura de inovação contínua e eficaz.

\subsection{Tecnologia}

Nesta dimensão, preocupou-se em aferir a capacidade da instituição em adotar e explorar as tecnologias para transformar seus processos e a prestação de seus serviços, além de facilitar o acesso à informação. As tecnologias emergem como um conjunto de recursos fundamentais que permitem o planejamento, a implantação e a integração de soluções eficazes capazes de apoiar o negócio digital Salume, Barbosa, Pinto e Sousa \cite{salume2021key}. 

O sucesso nesta dimensão é medido pela capacidade da instituição em implementar sistemas que não só modernizam as operações, mas também promovem a integração e colaboração interna e externa. A adesão a uma cultura orientada por dados e a busca por soluções tecnológicas inovadoras são indicativos da disposição do Ministério Público em evoluir e se manter relevante no cenário digital atual, destacando a importância de uma abordagem proativa na adoção de tecnologias para uma efetiva transformação digital.

Hipótese 8: O Ministério Público Brasileiro pode estar enfrentando desafios significativos na integração e otimização de tecnologias digitais em suas operações e estratégias. Isso pode ser devido à falta de sistemas totalmente integrados, à resistência à mudança, ou à insuficiência em desenvolver uma cultura orientada por dados e inovação. Como resultado, a instituição pode estar perdendo oportunidades vitais de aprimorar a transparência, agilidade e eficácia dos seus serviços no ambiente digital contemporâneo.
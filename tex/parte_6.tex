\section{Considerações Finais}

O objetivo deste trabalho foi o de estudar a existência de diferenças salariais e diferenças na quantidade de demissões por gênero no setor de TI no Brasil.

Observando a população feminina e a população masculina a partir dos dados que a RAIS nos fornece, foi possível perceber que a desigualdade salarial entre gêneros evidenciadas tanto na pesquisa de 2022 do IBGE quanto na pesquisa de 2015 da ONU, aqui já citadas,  são também percebidas entre os profissionais da área de Tecnologia da Informação (TI) em todo o território brasileiro em desfavor da mulher e independentes do nível escolar, do cargo de TI e do setor onde trabalham, se público ou privado. E essa é uma realidade em todos os anos de estudo, de 2015 a 2021.

Os dados também nos permitiu evidenciar que esta é uma profissão predominantemente masculina, com uma diferença superior a 70\% em todos os anos analisados. Já com relação à quantidade de demissões, não foi possível perceber uma diferença significativa entre os gêneros, execeto em 2021, quando a porcentagem de demissões dos homens foi de 35,82\% e das mulheres doi de 29,91\%.

Os dados estudados não nos permitem esclarecer as causas dessas diferenças, apenas nos ajudam a evidenciá-las. É, portanto, necessário aprofundar os estudos no sentido de identificar os fundamentos dessas disparidades para que possam ser apontadas possíveis providências socioeconômicas e socioeducativas no sentido de mudar essa realidade. 

Dos resultados obtidos, percebe-se que há escopo para políticas específicas de crescimento da participação feminina na carreira de TI, que é considerada uma das carreiras mais promissoras da atualidade. Maiores estudos devem sugerir também o quanto da atual segregação no setor de tecnologia é gerada pelo tratamento desigual e quanto é gerada pelas preferências individuais.





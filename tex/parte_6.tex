\section{Considerações finais e trabalhos futuros}

A investigação realizada neste estudo sobre a maturidade digital no Ministério Público Brasileiro revela um panorama complexo, onde avanços significativos coexistem com desafios notáveis. A análise das dimensões cruciais da maturidade digital demonstrou que, embora haja progressos em áreas como serviços ao cidadão e governança, aspectos como liderança, pessoas, tecnologia e operações ainda requerem desenvolvimento mais robusto e estratégico, sobretudo na tradução para uma priorização efetiva da agenda digital no Ministério Público.

Este estudo, portanto, não apenas mapeia o estado atual da maturidade digital no Ministério Público Brasileiro, mas também serve como um sinalizador para futuras direções e estratégias necessárias para uma transformação digital mais integrada e efetiva.

Uma das conclusões mais significativas deste estudo é a constatação de que a transformação digital vai além da simples adoção de tecnologias; ela exige uma mudança cultural e estrutural profunda dentro das organizações. Esta abordagem multifacetada, entretanto, possui na falta de alfabetização digital das lideranças o seu grande entrave, uma vez que além de serem instituições históricas e conservadoras, a falta de destreza digital dos líderes impõe um cenário pouco fértil a mudanças e evoluções.

Do ponto de vista acadêmico, este trabalho preenche uma lacuna importante na literatura sobre transformação digital no setor público brasileiro. Os achados oferecem uma base sólida para futuras pesquisas nesta área, destacando áreas-chave que necessitam de maior investigação e entendimento. Este estudo é particularmente relevante para gestores e decisores no setor público, oferecendo insights práticos e orientações estratégicas para navegar no complexo processo de transformação digital.

Contudo, este estudo não está isento de limitações, que, por sua vez, abrem portas para futuras pesquisas. Uma abordagem longitudinal poderia fornecer uma compreensão mais rica da evolução da maturidade digital ao longo do tempo. Além disso, métodos qualitativos, como estudos de caso e entrevistas, poderiam oferecer uma visão mais aprofundada das experiências e percepções individuais relacionadas à transformação digital. Outra direção frutífera para pesquisas futuras seria a expansão do escopo do estudo para incluir uma variedade mais ampla de organizações do setor público, tanto no Brasil quanto em um contexto internacional, para comparar e contrastar diferentes abordagens e experiências de transformação digital.

Por fim, investigar a sequência e a priorização no desenvolvimento das capacidades digitais pode oferecer diretrizes valiosas para as organizações em sua jornada de maturidade digital. Isso é especialmente relevante para o setor público, onde a transformação digital deve ser abordada de forma estratégica, com alto patrocínio das lideranças, que, para tanto, precisam ter uma melhor concepção do que é o mundo digital em constante transformações, ora vivenciado. Em resumo, este estudo não apenas contribui para o entendimento atual da maturidade digital no setor público brasileiro, mas também estabelece um caminho para futuras investigações e práticas nessa área vital.
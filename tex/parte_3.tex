\section{Metodologia}

A fase empírica da pesquisa consistiu em um questionário conduzido por meio de um formulário eletrônico, elaborado no LimeSurvey. Esse formulário, nomeado de "Mensuração de Destreza Digital" \footnote{Importante registrar que ao realizar o envio, a Comissão de Planejamento Estratégico optou por utilizar o termo destreza digital e não maturidade digital, em razão de que, ainda que sem a tecnicidade acadêmica, a percepção é a de que as lideranças ministeriais ainda atrelam o conceito de maturidade digital à capacidade das áreas de TI em operacionalizar as tecnologias, razão pela qual foi feita a seguinte orientação no Formulário de mensuração da destreza digital: Manual do Usuário - Versão 1.0 (2023): Isto é o que chamamos de Destreza Digital: a habilidade de uma organização em harmonizar a mentalidade digital com a maturidade digital. Maturidade e mentalidade digital são dois conceitos que, embora estejam relacionados com a adoção e integração das tecnologias digitais nas atividades diárias e nas práticas de negócios, possuem nuances e implicações distintas. Vamos detalhar cada um desses conceitos: 1. Mentalidade Digital :Refere-se à disposição e à atitude das pessoas em relação à adoção, uso e valorização das tecnologias digitais e da inovação contínua. • Características:
• Abertura ao novo, disposição para experimentar e aprender com erros. • Enxergar possibilidades e soluções por meio da tecnologia. • Colaboração e compartilhamento de informações. • Capacidade de se adaptar rapidamente a mudanças e transformações. • Pensamento crítico e questionador, buscando melhorias e inovações. • Implicações: Uma mentalidade digital forte pode levar a um ambiente onde as pessoas estão mais dispostas a se adaptar, inovar e superar desafios com o auxílio da tecnologia. 2. Maturidade Digital: Refere-se ao estágio em que uma organização ou indivíduo se encontra em termos de adoção, integração e otimização das tecnologias digitais em seus processos, estratégias e cultura. • Características: • Nível avançado de integração de tecnologias digitais nas operações diárias. • Processos e estratégias bem definidos para o uso da tecnologia. • Capacidade de medir e avaliar o impacto das iniciativas digitais. • Existência de governança digital, padrões e políticas para a gestão da tecnologia. • Cultura organizacional que apoia e promove a transformação digital. • Implicações: Organizações com alta maturidade digital geralmente têm melhor desempenho no mercado, são mais inovadoras e estão mais bem preparadas para enfrentar desafios e aproveitar oportunidades no ambiente digital.  Em sua essência, a destreza digital combina a abertura ao novo e a disposição para inovar com a capacidade efetiva de implementar, otimizar e governar iniciativas digitais; uma mentalidade digital pode impulsionar a busca por maior maturidade digital, enquanto uma alta maturidade pode reforçar e sustentar uma mentalidade digital positiva.} ficou disponível entre os dias 19 de outubro e 27 de novembro de 2023 e foi encaminhado pela CPE/CNMP, por meio de ofício, para os Procuradores-Gerais de Justiça de cada Ministério Público estadual e dos ramos do Ministério Público da União, o que corresponde a 30 unidades respondentes. Por esse motivo, nesta pesquisa não se utilizou uma metodologia para identificar o número de amostras para a realização da análise dos dados.

O instrumento de coleta de dados foi organizado em nove seções. A primeira seção contém questões relacionadas à identificação da unidade respondente e o responsável pela resposta. As demais seções correspondem às dimensões desenvolvidas por Rossmann (2018); para cada dimensão, uma seção no formulário, cada uma delas inclui perguntas com respostas objetivas “sim” ou “não”, ou com dados numéricos para informar quantitativo de recursos (estruturas, pessoal e orçamento). Vale ressaltar que os dados coletados em relação aos recursos de pessoal e orçamento correspondem aos anos de 2019 a 2022; o intuito foi de identificar a evolução da alocação desses recursos nos últimos 4 anos. Ademais, para garantir a autenticidade das respostas, muitas perguntas exigiram o envio de evidências reais, as quais podiam ser enviadas por meio de links ou arquivos anexados. Isso assegurou que as respostas fossem baseadas em práticas reais e não apenas teóricas.

Por fim, vale reforçar que a CPE, como guardiã da Estratégia Nacional do MP Digital, tem o objetivo de estabelecer diretrizes de governança e gestão que impulsionem o desenvolvimento, a coordenação, o planejamento, a priorização e a implementação de estratégias de inovação e fomento à evolução digital no Ministério Público. Para tanto, neste primeiro momento, o formulário serviu de instrumento para diagnosticar o cenário do MP brasileiro com relação às dimensões que capacitam as instituições a uma TD bem-sucedida. Por esse motivo, optou-se por não fazer uma pontuação para as perguntas/dimensões e classificar os ministérios em uma escala de maturidade. Nesta fase, o diagnóstico servirá de base para que a CPE compreenda as forças e fraquezas de cada unidade ministerial e elabore os planos e ações de atuação para conduzi-las a um grau de maturidade desejado. Nos tópicos subsequentes, serão apresentados os achados, as conclusões e as possibilidades de trabalhos futuros.

\section{Discussão dos Resultados}

A análise dos dados coletados do Ministério Público Brasileiro revelou nuances importantes na jornada nacional dos ramos e unidades para alcançar bons patamares na sua maturidade digital. As hipóteses formuladas inicialmente serviram como diretrizes para uma análise detalhada de cada dimensão, permitindo uma compreensão mais profunda dos desafios enfrentados pela instituição na adoção de práticas digitais. A divisão nas dimensões propostas com base no trabalho Rossmann \cite{rossmann2018digital} forneceu informações valiosas sobre as áreas em que o Ministério Público Brasileiro tem realizado progressos e onde se fazem necessários esforços adicionais.

Primeiramente, os resultados da dimensão Estratégia Digital confirmam a Hipótese 1, revelando uma realidade preocupante: apenas 2 das 29 unidades possuem uma estratégia digital definida, documentada, publicamente divulgada e que orienta efetivamente as ações de gestão. Esta constatação aponta para uma defasagem significativa na abordagem institucional à transformação digital, impactando diretamente o nível de maturidade digital do Ministério Público como um todo. A ausência de estratégias digitais formalizadas em grande parte das unidades do MP brasileiro indica que a transformação digital, até o momento, tem sido tratada de maneira fragmentada e inconsistente, o que também sugere que a adoção de tecnologias digitais e a remodelação de processos não estão sendo conduzidas de forma coordenada ou alinhada a um plano estratégico maior.

Do ponto de vista da maturidade digital, esta realidade revela que o Ministério Público brasileiro ainda está nos estágios iniciais de sua jornada de transformação digital. A falta de estratégias digitais coesas impede a instituição de aproveitar plenamente as vantagens que a digitalização oferece, limitando sua capacidade de responder às demandas de um ambiente cada vez mais digitalizado e conectado. Além disso, sem uma estratégia digital clara, torna-se desafiador para as unidades do MP estabelecerem prioridades, medirem progressos e adaptarem-se às rápidas mudanças tecnológicas e às expectativas dos cidadãos.

Em segundo lugar, a análise dos dados levantados referentes à dimensão Serviços ao Cidadão revela um cenário heterogêneo e com significativos desafios. Embora haja esforços para digitalização num patamar maior do que o esperado na hipótese 2, a oferta de serviços digitais integrados e abrangentes que atendam às diversas necessidades dos cidadãos ainda é pequena. Digno de nota é o fato de que o único serviço prestado por todas as unidades é aquele que possui exigência normativa do Conselho Nacional do Ministério Público - o serviço de acesso à ouvidoria -, enquanto outros serviços essenciais como a consulta de atendimentos e procedimentos, e o protocolo online para contato direto com órgãos de execução, apesar de ter presença significativa (86,7\% e 66,7\% respectivamente), ainda não são universais. No questionário, foram inseridos 9 serviços para o cidadão que a CPE identificou como essenciais aos MPs e passíveis de digitalização, mas o diagnóstico constatou que apenas 20\% das unidades possuem entre 0-3 serviços online e a maioria (66,67\%) oferece entre 4-7 serviços, com apenas 13,33\% disponibilizando 7 ou mais serviços, sublinha a necessidade urgente de expansão e integração digital. Ainda nesta seção, ficou assinalado que apenas metade dos Ministérios Públicos possui um catálogo de serviços, muitas vezes sem a distinção clara entre os serviços digitais e não digitais, o que aponta para uma deficiência na organização e na transparência da oferta de serviços. Um catálogo bem estruturado e atualizado é essencial para orientar os cidadãos e permitir um acesso fácil e eficiente aos serviços disponíveis. Finalmente, no tocante à avaliação da satisfação do usuário, embora 67\% das unidades tenham alguma ferramenta de avaliação, a maioria limita essa avaliação a serviços específicos. Isso sugere uma abordagem fragmentada que pode não captar a experiência do usuário de maneira abrangente. Avaliações de satisfação consistentes e abrangentes são fundamentais para medir a eficácia dos serviços, identificar áreas de melhoria e adaptar-se às necessidades e expectativas dos cidadãos. A ampliação da oferta de serviços online, a implementação de um catálogo de serviços completo e a adoção de avaliações de satisfação abrangentes são passos essenciais para fortalecer a relação de confiança e transparência com o público e para posicionar o Ministério Público como uma instituição ágil, acessível e alinhada com as demandas da era digital.

Em terceiro lugar, a análise da dimensão Pessoas no contexto da maturidade digital do Ministério Público Brasileiro revela uma realidade preocupante. Os dados indicam que a maioria dos MPs (63,33\%) não possui nenhum servidor dedicado integralmente à inovação e 27\% não possui sequer servidor com dedicação parcial nessa área. A situação é similar para as áreas de transformação digital e governança institucional. Já em governança de TI, esses números apresentam uma pequena melhora, mas ainda 50\% dos MPs não possui servidor com dedicação exclusiva e 40\% dizem ter entre 1 e 2 servidores com dedicação parcial ao tema. Esta lacuna na alocação de recursos humanos especializados é um indicador de que o Ministério Público está enfrentando desafios significativos para se adaptar e inovar no ambiente digital. A escassez de profissionais dedicados a inovação, transformação digital, governança institucional e de TI sugere uma limitação na capacidade de desenvolver e implementar estratégias digitais eficazes. Além disso, apenas 16,67\% dos MPs possuem programas de capacitação contínua em transformação digital e inovação para servidores; 50\% possuem programas de capacitação para os servidores de TI e 16,67\% possuem um programa abrangente de capacitação e alfabetização digital para todos os membros e servidores. Esses números apontam para uma deficiência crítica na preparação da força de trabalho para os desafios da era digital. Ainda nesta seção, identificou-se que apenas 1 MP possui um plano de carreira ou incentivos específicos para servidores de TI; essa deficiência não apenas limita o desenvolvimento profissional dos servidores existentes, mas também reduz a capacidade de reter os talentos existentes e de atrair novos talentos especializados, essenciais para impulsionar a transformação digital.

Em quarto lugar, os dados levantados em relação à dimensão Gestão e Governança mostram que 70\% dos MPs possui ações para automação dos processos de negócio e do uso de tecnologias digitais; 36,6\% monitoram o uso das ferramentas digitais disponibilizadas aos usuários internos; 40\% adotam metodologias ágeis na gestão de projetos institucionais, 40\% na gestão de projetos de TI e 90\% no desenvolvimento de sistemas. Bayu \cite{Hie_2019} afirma que "as novas formas de trabalhar no ambiente digital são equipes interfuncionais ágeis e dinâmicas, muitas vezes chamadas de 'processo scrum'" e os dados da pesquisa mostram que esta é uma realidade vivida apenas nas áreas de TI, mas que outras áreas precisam avançar no uso de metodologias ágeis. Ainda nesta dimensão, identificou-se que apenas 16,6\% os MPs não utilizam boas práticas de mercado (como ITIL) para fazer gestão dos serviços de TI.

Em quinto lugar, na dimensão Liderança, a pesquisa destacou um déficit significativo em termos de envolvimento e capacitação dos líderes em temas como inovação e TD. 63,3\% dos MPs responderam que membros da alta-administração já participaram de pelo menos um evento abordando um desses temas, mas que não há uma iniciativa mais contundente que fortaleça a compreensão, o engajamento e o apoio desses executivos no processo de TD. Esse fato é corroborado pelo seguinte dado: apenas 23,3\% do MPs adotaram algum tipo de tecnologia emergente (IA generativa, metaverso, realidade aumentada, blockchain) ou inovação aberta. Portanto, o quadro sugere a necessidade de um programa estruturado de alfabetização digital para os líderes, para que eles possam efetivamente guiar a organização através dos desafios do ambiente digital.

Em sexto lugar, na dimensão Operações, constatou-se que apenas 30\% dos MPs possuem estrutura organizacional focada na TD, mas não há pessoas alocadas com dedicação exclusiva nessa estrutura; a média percentual da quantidade de servidores de TI em relação ao total de usuários é de 3,73\%, ou seja, menos de 4 servidores de TI para 100 usuários; o gasto médio com o pessoal de TI equivale a 0,08\% do gasto com pessoal nas unidades ministeriais; em média, apenas 10,8\% do gasto com custeio nas unidades é dedicado ao custeio de TI; em média, apenas 0,47\% do orçamento dos MPs é destinado ao investimento em T, enquanto 2,8\% é dedicado a gastos com arquitetura e engenharia. Isso sinaliza a importância de um investimento mais substancial em tecnologia da informação e em recursos humanos especializados para apoiar adequadamente a digitalização.

Em penúltimo lugar, os dados relativos à Cultura Organizacional indicam desafios significativos na implementação de uma cultura adaptativa e inovadora, crucial para a maturidade digital. Apenas 23,3\% dos MPs relatam possuir um processo de gestão de mudança implementado, uma porcentagem preocupantemente baixa, considerando a velocidade e a escala das transformações digitais necessárias. Essa deficiência sugere que a maior parte dos MPs pode estar lutando para transitar efetivamente para novos paradigmas operacionais e tecnológicos, um aspecto fundamental na promoção de uma mudança cultural abrangente. Por outro lado, a divulgação e comunicação dos resultados das ações de transformação digital, realizada por 66,7\% dos MPs, é um aspecto positivo. Essa prática é essencial para manter a transparência, gerar engajamento interno e externo e reforçar a importância da transformação digital. No entanto, ainda existe um espaço considerável para melhorias, especialmente em termos de inclusão e abrangência dessas comunicações. A falta de programas estruturados para receber e tratar sugestões de mudanças e inovação em 40\% dos MPs é uma lacuna crítica. Apenas 16,7\% dos MPs possuem um processo bem definido para esta finalidade, o que limita significativamente a capacidade de inovação de baixo para cima e a participação ativa dos membros e servidores no processo de transformação digital. Este cenário aponta para uma oportunidade perdida de aproveitar insights valiosos e promover uma cultura de inovação inclusiva. A existência de iniciativas de premiação e reconhecimento para práticas inovadoras em 76,7\% dos MPs é um aspecto encorajador. Tais iniciativas são cruciais para motivar a inovação e reconhecer esforços bem-sucedidos, mas devem ser complementadas com estratégias mais abrangentes para efetivar uma mudança cultural profunda. A cooperação para o desenvolvimento e uso de sistemas com outros MPs, presente em 93,3\% dos casos, destaca uma forte inclinação para a colaboração interinstitucional. Este é um passo positivo para compartilhar recursos, conhecimentos e melhores práticas, fortalecendo a capacidade coletiva de adaptação e inovação. Esses dados desafiam a hipótese 7, sugerindo que o Ministério Público Brasileiro apesar de enfrentar desafios significativos na implementação de uma cultura organizacional adaptativa e inovadora, possui em boa medida práticas nesse sentido. A lacuna na gestão de mudanças, na estruturação de programas para sugestões de inovação e na comunicação eficaz dos resultados das transformações digitais indica uma necessidade maior de reformulação cultural a ser incentivada pelo MP Digital, a fim de maximizar o potencial de inovação e adaptabilidade em um ambiente tecnológico em rápida evolução, garantindo o desenvolvimento de uma cultura de inovação contínua e eficaz.

Por último, na dimensão Tecnologia, os dados mostram que 100\% dos MPs possuem sistemas de processo eletrônico, tanto para a área-fim como para a área-meio. Contudo, ainda falta investir na integração desses serviços com outros sistemas afins; apenas 3,33\% dos sistemas da área-meio se integram com o tramita.gov \footnote{Tramita.gov - \url{https://www.gov.br/economia/pt-br/assuntos/processo-eletronico-nacional/conteudo/tramita.gov.br}} (tramitação de processos administrativos eletrônicos) e 30\% dos sistemas da área-fim se integram com o Sistema Eletrônico de Execução Unificado - SEEU \footnote{SEEU - \url{https://seeu.pje.jus.br/seeu/}}. Ainda nesta dimensão, constatou-se que 93,3\% do MPs fornece algum painel de Business Inteligence para as áreas meio e fim, permitindo o fortalecimento da cultura baseada em dados, mas apenas 36,7\% desses MPs monitoram o uso efetivo desses painéis, portanto, não há como determinar a sua efetiva utilização pelos usuários. 43,3\% das unidade ministeriais fazem análise de qualidade dos seus dados e um número ainda menor, 30\%, possuem um dicionário de dados. Por fim, contatou-se que 50\% dos MPs têm utilizados tecnologia low code/no code em seus projetos de desenvolvimento.

\section{Introdução}

A era digital representa um avanço significativo na evolução socioeconômica humana. Sucedendo as revoluções tecnológicas primitivas e industriais, esta fase se caracteriza pela transformação da informação. Uma virada que pode ser constatada pelo fato de que, no final dos anos 1980, menos de 1\% da informação armazenada era digital; em 2012, esse número ultrapassou 99\%. Segundo Hilbert \cite{hilbert2020digital}, a cada 2,5 a 3 anos, a humanidade é capaz de armazenar mais informação do que desde o início da civilização e a era atual se concentra em algoritmos que automatizam a conversão de dados em conhecimento acionável. Essa realidade tem impactado todas as organizações, sejam elas públicas ou privadas.

Tal conjuntura tem exigido das organizações uma série de medidas que, a par da não unanimidade sobre sua definição, podem ser referidas como transformação digital, definida por Swen Nadkarni \& Reinhard Prugl \cite{nadkarni2021digital} como “o processo de mudança organizacional que envolve o uso de tecnologias digitais para possibilitar novas formas de criação de valor, interação com os clientes, operação dos processos e gestão dos recursos. A transformação digital requer uma visão estratégica, uma cultura de inovação, uma liderança transformadora e uma capacidade de adaptação contínua às mudanças tecnológicas e de mercado.” Essas medidas representam uma mudança substancial na maneira como as organizações funcionam, se comunicam e geram resultados, incorporando automação, robotização e digitalização em seus processos. Esse movimento engloba o emprego de tecnologias emergentes como computação em nuvem, análise de dados, inteligência artificial e a Internet das Coisas (IoT). 

Contudo, apesar da importância inegável dessa transição, as organizações enfrentam desafios significativos para se adaptarem a esta nova realidade digital. Isso requer mudanças estruturais profundas, com ênfase na centralização da tecnologia em suas operações. Tal mudança demanda uma nova mentalidade dos executivos e a formulação de uma Estratégia Digital robusta, pois "a capacidade de reimaginar digitalmente o negócio é determinada em grande parte por uma estratégia digital clara apoiada por líderes que fomentam uma cultura capaz de mudar e inventar o novo" \footnote{Gartner, 2021 \url{https://www.cogef.ms.gov.br/wp-content/uploads/2021/10/Transformacao-Digital-Gartner.pdf}}.

Bharadwaj, Sawy, Pavlou, \& Venkatraman \cite{bharadwaj2013digital} definem Estratégia Digital como “estratégia organizacional formulada e executada por meio do aproveitamento de recursos digitais para criar valor diferencial”. Em outras palavras, uma estratégia digital define como os recursos digitais disponíveis, tais como as tecnologias de informação e comunicação, as plataformas digitais, os dados, a inteligência artificial, a nuvem etc. serão utilizados para apoiar o alcance dos objetivos e metas organizacionais. A estratégia digital envolve uma mudança na forma de pensar, de agir e de se relacionar com o ambiente externo e interno da organização, aproveitando as oportunidades e os desafios que o contexto digital oferece; inclui o envolvimento dos usuários e dos funcionários, a criação de equipes multidisciplinares e ágeis, o estímulo à experimentação e à aprendizagem, o incentivo à cooperação e à cocriação, e a garantia da segurança e da ética. Uma estratégia digital serve para orientar e otimizar as ações de transformação digital, de modo a garantir a eficiência, a qualidade e a satisfação dos clientes com os produtos e serviços oferecidos.

No setor público, a resistência a mudanças é ainda mais evidente, o que torna a Transformação Digital (TD), especialmente em instituições como o Ministério Público, uma tarefa complexa e desafiadora, mas há uma dificuldade ainda maior que a observação realizada com este projeto parece evidenciar: a falta de Alfabetização Digital (AD) entre muitos líderes dessas instituições. Essa lacuna de conhecimento digital nas lideranças se manifesta como um obstáculo significativo, impedindo-os de definir e implementar estratégias digitais eficazes. 

A urgência na transformação digital no setor público reforça a importância de líderes bem-informados e capacitados digitalmente. Doutra maneira, as lideranças precisam compreender melhor não apenas as tecnologias existentes, mas também suas aplicações e potenciais impactos nas operações e na cultura organizacional, conditio sine qua non para o desenvolvimento de uma mentalidade digital que, por sua vez, é indispensável para o alcance da transformação, segundo Morakanyane et al., 2017 \cite{bharadwaj2013digital}.

Em pesquisa realizada pela Boston Consulting Group \footnote {BCG \url{https://www.bcg.com/publications/2020/increasing-odds-of-success-in-digital-transformation}}, em 2020, com 70 grandes empresas e 825 executivos seniores como respondentes, foi constatado que 70\% das empresas falham na sua TD devido às barreiras organizacionais (rodapé BCG), as quais podem ser resumidas como sendo uma inércia organizacional de comportamentos profundamente enraizados e uma cultura resistente à mudanças \cite{vial2021understanding}, reflexos dos resultados organizacionais positivos atingidos por meio de um modelo de negócio tradicional e vigente. Diante do exposto, Morakanyane et al. \cite{bharadwaj2013digital}, 2017 diz que "para se prosperar em uma jornada de TD, a organização precisa de um conjunto de competências, mentalidade e cultura que sejam digitais".

Com o intuito de evidenciar esses e outros pontos que merecem a atuação na jornada de TD do Ministério Público Brasileiro, a Comissão de Planejamento Estratégico (CPE) do Conselho Nacional do Ministério Público (CNMP), sob a égide da Estratégia Nacional do MP Digital \footnote{Estratégia Nacional do MP Digital \url{https://www.cnmp.mp.br/portal/atos-e-normas/norma/9724/}}, conduziu uma pesquisa empírica junto às unidades ministeriais de cada estado brasileiro e aos ramos do Ministério Público da União. Para a realização deste trabalho, foi efetuada uma vasta consulta na literatura nacional e internacional a respeito do tema. Dentre os trabalhos aqui referenciados, ressalta-se dois, o de Rossmann \cite{rossmann2018digital} e o de Salume, Barbosa, Pinto e Sousa \cite{salume2021key}. No primeiro, foi realizada uma compilação abrangente de vários estudos visando desenvolver um modelo de avaliação para a maturidade digital, o qual, resumidamente, consiste em oito áreas essenciais: estratégia, liderança, mercado, operações, recursos humanos, cultura organizacional, governança e tecnologia. O segundo trabalho consiste na aplicação do modelo sugerido por Rossmann \cite{rossmann2018digital} para a condução de uma pesquisa empírica no setor de varejo do Brasil com o objetivo de apontar quais elementos ou dimensões estão relacionados ao desenvolvimento da maturidade digital.

Vale pôr em evidência que a literatura sobre TD no Brasil ainda é incipiente. Uma pesquisa no portal da Scientific Periodicals Electronic Library (Spell), mantido pela Associação Nacional de Pós-Graduação e Pesquisa em Administração (Anpad), sob o rótulo “Maturidade Digital”, realizada em dezembro de 2023, retornou apenas um trabalho, o já aqui referenciado Salume, Barbosa, Pinto e Sousa (2021). Outra pesquisa, sob o rótulo “Transformação Digital”, realizada na mesma data, retornou 47 trabalhos, mas apenas dois focados no setor público, um realizado por Brognoli \cite{da2020transformaccao}, cujo objetivo era identificar os desafios, ações e perspectivas do governo brasileiro para a TD e o outro de ORLANDI, que mostrou o efeito positivo da TD como ferramenta simples e moderna de fiscalização, propiciando economia e eficiência ao controle público.

---
